\documentclass[a4paper, 11pt]{article}
\usepackage[T2A]{fontenc}
\usepackage[utf8]{inputenc}
\usepackage[russian]{babel}
\usepackage{indentfirst}
\usepackage{amssymb}
\usepackage{enumitem}
\usepackage{hyperref}
\usepackage{geometry}
\usepackage{mathtools}
\usepackage{setspace}
\usepackage{amsmath,amsfonts,amssymb,amsthm,mathtools}
\usepackage{tikz}
\usepackage{tikzsymbols}
\usepackage{soul}
\geometry{a4paper,top=2cm,bottom=2cm,left=2cm,right=2cm}

\usepackage{fancyhdr}
\pagestyle{fancy}

\makeatletter
\def\thickhrulefill#1{\leavevmode\leaders\hrule height#1\hfill\kern\z@}
\makeatother

\usepackage{mathtools}
\DeclarePairedDelimiter\ceil{\lceil}{\rceil}
\DeclarePairedDelimiter\floor{\lfloor}{\rfloor}

% Розовый текст в рамке
\newcommand{\pinkframed}[2][rectangle,draw,fill=purple!20]{%
	\tikz[baseline=-0.6ex] \node [#1]{#2};}%

\newenvironment{result} { \sgap
	\smallskip
	\pinkframed{ \textbf{Ответ:} }}{}

\renewcommand{\headrule}{
	\color{black}\vspace{-8pt}
	\thickhrulefill{.6pt}
	
	\vspace{-9pt}\thickhrulefill{2.4pt}
}

\newcommand{\un}{\underline}
\newcommand{\ov}{\overline}


\newcommand{\NN}{\mathbb{N}}
\newcommand{\ZZ}{\mathbb{Z}}
\newcommand{\RR}{\mathbb{R}}
\newcommand{\QQ}{\mathbb{Q}}
\newcommand{\CC}{\mathbb{C}}

\newcommand{\A}{\; \forall \;}
\newcommand{\E}{\; \exists \;}
\newcommand{\nE}{\; \nexists \;}

\newcommand{\vp}{\varphi}
\newcommand{\e}{\varepsilon}
\newcommand{\lm}{\lambda}
\newcommand{\ds}{\displaystyle}

\newcommand{\prob}[1]{\item \textbf{(#1 баллов)}.}


\rhead{Парфенов Артем БПМИ219}
\lhead{Основы матричных вычислений ДЗ-3}
\begin{document}
	

\begin{enumerate}
	\prob{20} Пусть задана матрица $A\in\mathbb{R}^{m\times n}$, $m\geq n$. 
	\begin{enumerate}
		\item Покажите, что $A$ можно привести к верхнетреугольной матрице $R$ с помощью преобразований Хаусхолдера, используя 
		\[ 
		2mn^2 - \frac{2}{3} n^3 + \mathcal{O}(mn),
		\]
		арифметических операций. 
		
			$H(V) = I - 2VV^T$
			
			суть преобразований состоит в последовательном умножении матрицы $A$ на $H_i$ - матрицы Хаусхолдера.
			
			Посмотрим на первое умножение:
			
			$HA = (I - 2VV^T)A = A - v(2v^TA)$
			
			посчитаем операции:
			
			$v^TA $ потребует $2mn - n$ операций
			
			умножение на 2 вектора потребует $n$ операций, разность с матрицей $A$ еще $mn$ операций. Итого к этому моменту имеем $4mn$.
			
			На $i$"=той итерации будет 
			
			$H_i = \begin{pmatrix}
				I_k & 0 \\
				0 & H(V_k) \\ 
			\end{pmatrix}$
		
			потребуется $4(m - i)(n - i) = 4mn - 4ni - 4mi + 4i^2$
			
			Тогда итоговую сложность можно найти:
			
			$\displaystyle \sum_{i = 0}^{n - 1} (4mn - 4in - 4im + 4i^2) = 4mn^2 - 4n \cdot \binom{n}{2} - 4m \cdot \binom{n}{2} + 4 \cdot \frac{(n - 1) \cdot (n - 2) \cdot (2n - 3)}{6} = 2mn^2 - \frac{2n^3}{3} + O(mn)$
		
		\item Покажите, что количество арифметических операций для вычисления $Q\in\mathbb{R}^{m\times n}$ из тонкого QR будет:
		\[ 
		2mn^2 - \frac{2}{3} n^3 + \mathcal{O}(mn).
		\]
		
			$Q = H_1 \dots H_n \cdot \begin{pmatrix}
				I \\
				0 \\
			\end{pmatrix}$
		
		$\hat{H} =  H \cdot \begin{pmatrix}
			I \\
			0 \\
		\end{pmatrix}$
	
		На $i$"=той итерации $H_{n - i} = \begin{pmatrix}
			I_{n - i} & 0 \\
			0 & H(V_{n - i}) \\
		\end{pmatrix}$ уйдет $4(m - n + i) \cdot i$ из тех же соображений, что и в предыдущем пункте.
	
		Итого:
		
		$\displaystyle \sum_{i = 1}^{n} 4mi - 4ni + 4i^2 = 2mn^2 - \frac{2n^3}{3} + O(mn)$
		
			
	\end{enumerate}
	
	\prob{20} Запишем решение $x_\mu$ задачи наименьших квадратов с $\ell_2$-регуляризацией
	\[
	\|Ax - b\|_2^2 + \mu \|x\|_2^2 \to \min_x
	\]
	для заданной матрицы $A\in\mathbb{R}^{m\times n}$ ранга $r$, вектора правой части $b\in\mathbb{R}^{m\times n}$ и константы $\mu\in\mathbb{R}$ в виде
	$x_\mu = B(\mu) b$
	с матрицей $B(\mu)\in \mathbb{R}^{n\times m}$, которая выражается через $A$ и $\mu$ (см. лекцию).
	\begin{enumerate}
		\item Покажите, что для $\mu>0$ справедливо:
		\[
		\|B(\mu) - A^+\|_2 = \frac{\mu}{\left(\mu + \sigma_r(A)^2\right)\sigma_r(A)}.
		\]
		
			$\|B(\mu) - A^+\|_2 = \|(A^TA + \mu I)^{-1} A^T - A^+\|_2 = \|(V \Sigma^T \Sigma V^T   +\mu V V^T )^{-1} V \Sigma U^T - V \Sigma^+ U^T \|_2 = \| V (\Sigma^2 \mu I )^{-1}  V^{-1}V \Sigma U^T - V \Sigma^+ U^T \|_2 = \|V ((\Sigma^2  + \mu I)^{-1} \Sigma U^T - \Sigma^+ U^T)\|_2 = \|(\Sigma^2 + \mu I)^{-1} \Sigma - \Sigma^+\|_2$ - эта штука будет страшим сингулярным числом. 
			
			Обращать скобу можно, там определитель ненулевой.
			
			$Sigma^2 + \mu I = \begin{pmatrix}
				\sigma_1^2  + \mu & \dots & 0 \\
				\vdots & \dots & \vdots \\
				0 & \dots & \mu \\
			\end{pmatrix}$
		
			$(\Sigma^2 + \mu I)^{-1} \Sigma - \Sigma^+ = \begin{pmatrix}
				\frac{\sigma_1}{\sigma_1^2 + \mu} - \frac{1}{\sigma_1} & \dots \\
				\vdots & \dots \\
				0 & \dots & 0 \\
			\end{pmatrix}$
		
			Тогда $\|(\Sigma^2 + \mu I)^{-1} \Sigma - \Sigma^+\|_2 = \frac{\mu}{(\sigma_1^2 + \mu) \sigma_r}$
		
		\item Покажите, что $B(\mu)\to A^+$ и что $x_\mu\to A^+b$ при $\mu\to +0$.
		
			$B(\mu) \to A^+$ as $\mu \to +0 \Longleftrightarrow \lim\limits_{\mu \to x +0} \|B(\mu) - A^+\|_2 = 0$
			
			$\lim\limits_{\mu \to x +0} \|B(\mu) - A^+\|_2 = \lim\limits_{\mu \to x +0} \frac{\mu}{(\sigma_r^2 + \mu) \sigma_r} = 0$
			
			$\lim\limits_{\mu \to x +0} \|B(\mu)b - A^+b\|_2 = \lim\limits_{\mu \to x +0} \|(B(\mu) - A^+)b\|_2 = \lim\limits_{\mu \to x +0} \|0 b\|_2 = 0$
			
			успех.
		
	\end{enumerate}
	\prob{15} Покажите, что для решений $x\in\mathbb{R}^n$ задачи $\|Ax - b\| \to \min_x$, где $A\in\mathbb{R}^{m\times n}$, $b\in\mathbb{R}^{m}$ заданы, справедливо:
	\[
	\|x\|_2^2 = \|A^+b\|_2^2 + \|(I - A^+ A)y\|_2^2, 
	\]
	где $y$ -- произвольный вектор (см. обозначения в лекции). Сделайте отсюда вывод, какое решение имеет наименьшую $\|x\|_2$. 
	
		$\|x\|_2^2 = \|A^+ b\|_2^2 + \|(I - A^+A)y\|_2^2$
		
		тк из лекций знаем, что минимум второй нормы будет при $x = A^+b$
		
		тогда $\exists t_1, t_2 \in \RR^n :  x = t_1 + t_2, \langle t_1, t_2 \rangle = 0$
		
		$\sqsupset t_2 = (I - A^+A)y \in Ker(A)$
		
		$A^+b = V \Sigma^+ U^T b = V (\Sigma^+ U^T 	b) \in Im(V) = Im(A^T)$
		
		тогда $\langle A^+b, (I - A^+A)y \rangle = 0$
		
		опять успех.
	
	\prob{25}
	Пусть ненулевые $a, b\in\mathbb{R}^{n}$, $n\geq 2$ ортогональны друг другу и 
	\[
	A = a \circ a \circ a + 2 (a \circ b \circ a) + 3 (b \circ b \circ a).
	\]
	\begin{enumerate}
		\item Запишите матрицы $U,V,W \in\mathbb{R}^{n\times 2}$ из канонического разложения $A$. 
		\\ \textbf{Подсказка:} используйте линейность тензорного произведения по каждому из  аргументов.
		
			$A = a \circ a \circ a + 2 (a \circ b \circ a) + 3 (b \circ b \circ a) = a \circ (a \circ a + 2b \circ a) + 3b \circ b \circ a = a \circ (a + 2b) \circ a + 3b \circ b \circ a$
			
			$U = \begin{bmatrix}
				a & 3b \\
			\end{bmatrix}, V = \begin{bmatrix}
			a + 2b & b \\
			\end{bmatrix}, W = \begin{bmatrix}
				a & a \\
			\end{bmatrix}$
			
		\item Запишите ядро $G \in\mathbb{R}^{2\times 2 \times 1}$ и факторы $U,V,W$ из разложения Таккера $A$.
		
			$A = a \circ a \circ a + 2 (a \circ b \circ a) + 3 (b \circ b \circ a)$
			
			$G = \begin{bmatrix}
				1 & 2 \\
				0 & 3 \\
			\end{bmatrix}$
		
			$U = \begin{bmatrix}
				a & b \\
			\end{bmatrix}, V = \begin{bmatrix}
			a & b \\
				\end{bmatrix}, W = \begin{bmatrix}
				a \\
			\end{bmatrix}$
		
		\item Докажите, что мультилинейный ранг тензора $A$ равен $(2, 2, 1)$.
		
			Можно узнать ранг из развертки;
		
			$A_{(1)} = U G_{(1)} (W \otimes V)^T = \begin{bmatrix}
				a & b \\
			\end{bmatrix} \begin{bmatrix}
			1 & 2 \\
			0 & 3 \\
		\end{bmatrix} \begin{bmatrix}
		a \otimes \begin{bmatrix}
			a & b \\
		\end{bmatrix} \\
	\end{bmatrix}^T$
			
			$a \otimes \begin{bmatrix}
				a & b \\
			\end{bmatrix} = \begin{pmatrix}
			a_1 a & a_1 b \\
			a_2 a & a_2 b \\
			\vdots & \vdots \\
			a_n a & a_n b \\
		\end{pmatrix}$ 
		
		
		тогда 
		
		$rk A_{(1)} = 2$ тк ранги всех сомножителей равны двум
		
		
		$A_{(2)} = V G_{(1)} (W \otimes U)^T = \begin{bmatrix}
			a & b \\
		\end{bmatrix} \begin{bmatrix}
			1 & 0 \\
			2 & 3 \\
		\end{bmatrix} \begin{bmatrix}
			a \otimes \begin{bmatrix}
				a & b \\
			\end{bmatrix} \\
		\end{bmatrix}^T$
	
		$rk A_{(2)} = 2$ аналогично 
		
		
		$A_{(1)} = W G_{(1)} (V \otimes U)^T = \begin{bmatrix}
			a  \\
		\end{bmatrix} \begin{bmatrix}
			1023 \\
		\end{bmatrix} \begin{bmatrix}
			\begin{bmatrix}
				a & b \\
			\end{bmatrix} \otimes \begin{bmatrix}
				a & b \\
			\end{bmatrix} \\
		\end{bmatrix}^T$
	
		$rk A_{(3)} = 1$ тк мы умножаем столбец на строку.
			
	\end{enumerate}
	
	\prob{20}
	Пусть $A,B\in\mathbb{R}^{n\times n}$ -- некоторые заданные матрицы, и пусть стоит задача вычисления матрично-векторного произведения:
	\[
	y = \left(A\otimes B \right) x, \quad x \in\mathbb{R}^{n^2}.
	\]
	\begin{enumerate}
		\item Каково асимптотическое число арифметических операций для вычисления $y$ по $x$ без учета дополнительной структуры матрицы $\left(A\otimes B \right)$?
		
			в $A \otimes B$ будет $n^2$ строк, на подсчет каждой в произведении с иксом тоже потребуется порядка $n^2$. Тк просят асимптотику, можно забить на константу. 
			
			Общая сложность будет $O(n^4)$
		
		\item Предложите алгоритм вычисления $y$, имеющий асимптотическое число операций $\mathcal{O}(n^3)$. \textbf{Подсказка:} в этой задаче может помочь операция векторизации.
		
			$(A \otimes B) x = \begin{pmatrix}
				a_{11} B & \dots & a_{1n} B \\
				\vdots & \dots & \vdots \\
				a_{n1} B & \dots & a_{nn} B\\
			\end{pmatrix} \begin{pmatrix}
			x_{11} \\
			\vdots \\
			x_n^2 \\
		\end{pmatrix} = \begin{pmatrix}
		B_{(1)} X^{(1)} & \dots & B_{(1)} X^{(n)} \\
		\vdots & \dots & \vdots \\
		B_{(n)} X^{(1)} & \dots& B_{(n)} X^{(n)} \\
	\end{pmatrix} \cdot \begin{pmatrix}
	a_{11} & \dots & a_{1n} \\
	\vdots & \dots & \vdots \\
	a_{1n} & \dots & a_{nn} \\
\end{pmatrix} =  \begin{pmatrix}
	b_{11} & \dots & b_{1n} \\
	\vdots & \dots & \vdots \\
	b_{n1} & \dots & b_{nn} \\
\end{pmatrix} \cdot \begin{pmatrix}
	x_{11} & \dots& x_{1n} \\
	\vdots & \dots & \vdots \\ 
	x_{n1} & \dots& x_{nn} \\
\end{pmatrix} \cdot \begin{pmatrix}
	a_{11} & \dots & x_{n1} \\
	\vdots & \dots & \vdots \\
	a_{1n} \dots & a_{nn} \\
\end{pmatrix} = BXA^T$


		Считаем сложность. $\forall i, j B_{(i)} X^{(j)}$ требует $2n - 1$ операций для каждого из $n^2$ умножений. 
		
		$BXA^T$ из аналогичных соображений требует столько же в асимптотике, а именно $O(n^3)$
		
		\item Как получить число операций $\mathcal{O}(n^2\log n)$, если $A$ и $B$ являются циркулянтами?
		
			$BXA^T = (A(BX)^T)^T$
			
			Теперь заметим, что $A^T$ будет циркулянтом, тк $A$ такова.
			
			$(A(BX)^T)^{(i)}$ можно посчитать за $O(n \log n)$ тк мы умножаем вектор на циркулянт. 
			
			Таких умножений требуется $n$ штук, итого имеем сложность $O(n^2 \log n)$
		
	\end{enumerate}
\end{enumerate}


\subsection*{Бонусные задачи}
\begin{enumerate}
	\item \textbf{(40 б. балла)}. 
	Пусть $A\in\mathbb{C}^{m\times n}$, $m \geq n$ -- матрица полного ранга. Верно ли, что в ее тонком $QR$-разложении все $r_{kk}$, $k=1,\dots,n$ (диагональные элементы $R$) всегда можно выбрать вещественными положительными? Пусть $r_{kk}>0$, $k=1,\dots,n$. Покажите, что в таком случае разложение определяется единственным образом. 
	\item \textbf{(60 б. балла)}. 
	Докажите или опровергните, что 
	\[
	\|X^+X - Y^+ Y\|_F \leq \left( \|X^+\|_2 + \|Y^+\|_2 \right) \|X - Y\|_F,
	\]
	для любых комплексных матриц $X$ и $Y$ одного размера. \\
	\textbf{Подсказка:} подумайте над тем, что можно добавить и вычесть под знаком нормы.
\end{enumerate}
	
	
\end{document}
\documentclass[a4paper, 11pt]{article}
\usepackage[T2A]{fontenc}
\usepackage[utf8]{inputenc}
\usepackage[russian]{babel}
\usepackage{indentfirst}
\usepackage{amssymb}
\usepackage{enumitem}
\usepackage{hyperref}
\usepackage{geometry}
\usepackage{mathtools}
\usepackage{setspace}
\usepackage{amsmath,amsfonts,amssymb,amsthm,mathtools}
\usepackage{tikz}
\usepackage{tikzsymbols}
\usepackage{soul}
\geometry{a4paper,top=2cm,bottom=2cm,left=2cm,right=2cm}

\usepackage{fancyhdr}
\pagestyle{fancy}

\makeatletter
\def\thickhrulefill#1{\leavevmode\leaders\hrule height#1\hfill\kern\z@}
\makeatother

\usepackage{mathtools}
\DeclarePairedDelimiter\ceil{\lceil}{\rceil}
\DeclarePairedDelimiter\floor{\lfloor}{\rfloor}

% Розовый текст в рамке
\newcommand{\pinkframed}[2][rectangle,draw,fill=purple!20]{%
	\tikz[baseline=-0.6ex] \node [#1]{#2};}%

\newenvironment{result} { \sgap
	\smallskip
	\pinkframed{ \textbf{Ответ:} }}{}

\renewcommand{\headrule}{
	\color{black}\vspace{-8pt}
	\thickhrulefill{.6pt}

	\vspace{-9pt}\thickhrulefill{2.4pt}
}

\newcommand{\un}{\underline}
\newcommand{\ov}{\overline}


\newcommand{\NN}{\mathbb{N}}
\newcommand{\ZZ}{\mathbb{Z}}
\newcommand{\RR}{\mathbb{R}}
\newcommand{\QQ}{\mathbb{Q}}
\newcommand{\CC}{\mathbb{C}}

\newcommand{\A}{\; \forall \;}
\newcommand{\E}{\; \exists \;}
\newcommand{\nE}{\; \nexists \;}

\newcommand{\vp}{\varphi}
\newcommand{\e}{\varepsilon}
\newcommand{\lm}{\lambda}
\newcommand{\ds}{\displaystyle}

\newcommand{\prob}[1]{\item \textbf{(#1 баллов)}.}


\rhead{Парфенов Артем БПМИ219}
\lhead{Основы матричных вычислений ДЗ-2}
\begin{document}

\textcolor{blue}{Во всех задачах считайте, что $m,n\geq 2$.}
\begin{enumerate}
	\prob{12} Найдите скелетное разложение вида $C\widehat A^{-1} R$ матрицы $m\times n$  с элементами: \[a_{ij} = \frac{i}{j} + \frac{j}{i}.\] Нумерация индексов начинается с $1$.

		Матрица имеет вид: $\begin{pmatrix}
			2 && 2 + \frac{1}{2} && \dots && n + \frac{1}{n} \\
			2 + \frac{1}{2} && 2 && \dots && \frac{2}{n} + \frac{n}{2} \\
			\vdots && \vdots && \dots && \vdots \\
			m + \frac{1}{m} && \frac{m}{2} + \frac{2}{m} && \dots && \frac{m}{n} + \frac{n}{m} \\
		\end{pmatrix} = \begin{pmatrix}
		1 && \frac{1}{2} && \dots && \frac{1}{n} \\ 
		2 && 1 && \dots && \frac{2}{n} \\
		\vdots && \vdots && \dots && \vdots \\
		m && \frac{m}{2} && \dots && \frac{m}{n} \\
	\end{pmatrix} + \begin{pmatrix}
		1 && 2 && \dots && n \\
		\frac{1}{2} && 1 && \dots && \frac{n}{2} \\
		\vdots && \vdots && \dots && \vdots \\
		\frac{1}{m} && \frac{2}{m} && \dots && \frac{n}{m} \\
		\end{pmatrix}$
	
	 ранги двух матриц - слагаемых равны единиц, тк их можно представить в виде произведения двух векторов (от 1 до m и от 1 до 1/n), таким образом ранг суммы не превосходит двух.
	 
	 

		\begin{enumerate}
			\item 	Хотим узнать ранг, для этого посчитаем миноры. Не помню, как они обозначались в линале, договоримся, что буквой t.

			$t_0 = 2 \neq 0, t_1 = 4 - 2(2 + \frac{1}{2}) \neq 0$

			$t_2 = 0$, значит старший ненулевой минор - второй(индекс равен 1), откуда ранг матрицы равен 2. Потому что второй минор ненулевой, значит - ранг не меньше 2.

			\item Теперь строим CUR-разложение:

				Раз знаем, что ранг 2, значит возьмём верхнюю левую подматрицу в качестве $\widehat A$

				$C\widehat A^{-1} R = \begin{pmatrix}
					2 && 2 + \frac{1}{2} \\
					2 + \frac{1}{2} && 2 \\
					\vdots && \vdots \\
					m + \frac{1}{m} && \frac{m}{2} + \frac{2}{m} \\
				\end{pmatrix} \cdot \begin{pmatrix}
				\frac{-8}{9} & \frac{10}{9} \\
				\frac{10}{9} & \frac{-8}{9}
			\end{pmatrix} \cdot \begin{pmatrix}
				2 && 2 + \frac{1}{2} && \dots && n + \frac{1}{n} \\
				2 + \frac{1}{2} && 2 && \dots && \frac{2}{n} + \frac{n}{2} \\
		\end{pmatrix}$

		\end{enumerate}




	\prob{15}
	Пусть $S,S_1, S_2\subset \mathbb{R}^n$, а $S^\perp,S_1^\perp, S_2^\perp$ -- их ортогональные дополнения.
	\begin{enumerate}
		\item Покажите, что $\mathrm{dist}(S_1,S_2) = \mathrm{dist}(S^\perp_1,S^\perp_2)$.

			$\square$

			$\mathrm{dist}(S_1,S_2) = \|P_1 - P_2 \|_2$, где $P_1, P_2$ - соответствующие ортопроекторы.

			Знаем, что $\|-t\|_2 = \| t\|_2$

			$\mathrm{dist}(S^\perp_1,S^\perp_2) = \|(I - P_1) - (I - P_2)\|_2$ из свойств ортопроектора

			$= \|P_2 - P_1\|_2 = \|-(P_1 - P_2)\|_2 = \|P_1 - P_2 \|_2 = \mathrm{dist}(S_1,S_2) \hfill \blacksquare$

		\item Найдите $\mathrm{dist}(S,S^\perp)$.

			$\|P - I + P \|_2 = \|2P - I \|_2 = \lambda_{max}((2P - I)^*(2P - I))$ звезда - это эрмитово сопряжение

			$(2P - I)^*(2P - I) = 4P^*P - 2P^* - 2P + I = 4P^2 - 4P + I$ (тк P - ортопроектор)

			$= 4P - 4P + I = I$ (по тем же причинам)

			$(I - \lambda I) = 0$

			$\lambda = 1$ это я типа харчлен написал.

			Ну все, лямбда равна единице, значит и вторая норма, значит и дист.

	\end{enumerate}
	\prob{15}
	Пусть $U = [U_r\ U_r^\perp ]\in\textcolor{blue}{\mathbb{C}^{m\times m}}$ -- матрица левых сингулярных векторов матрицы $A\in \mathbb{C}^{m\times n}$ ранга $r$. Покажите, что $\mathrm{ker}(A^*) = \mathrm{Im}(U_r^\perp)$ и запишите ортопроектор на \textcolor{blue}{$\mathrm{ker}(A^*)$}.
	
	
		$\square$

		\begin{enumerate}
				\item $x \in Im(U_r^\perp) \implies x \in \mathrm{ker}(A^*) \Longleftrightarrow Im(U_r^\perp) \subseteq \mathrm{ker}(A^*)$

				$\forall x \in Im(U_r^\perp) \implies \exists U_r^\perp t = x \implies A^* x = V \Sigma^* U^* U \begin{pmatrix}
					0 \\
					t \\
				\end{pmatrix} = V \Sigma^* I \begin{pmatrix}
				0 \\
				t \\
			\end{pmatrix} = 0$

			тк сигма состоит из диагонального блока слева вверху, а вектор t в этом месте имеет нули, после произведения получится так.

			\item	$ \mathrm{ker}(A^*) \subseteq   Im(U_r^\perp)$
			
				$\sqsupset x \in \mathrm{ker}(A^*)$
				
					$A^* x = V \Sigma^* U^* x = 0$
					
					$\begin{pmatrix}
						v_1 \bar \sigma_1 && \dots && v_r \bar \sigma_r && 0  \\ 
					\end{pmatrix} \begin{pmatrix}
					u_r^* \\ 
					u_r^{\perp*} \\
				\end{pmatrix} x =$
			
				назовем все кроме первой матрицы m
				
				$= v_1 \bar \sigma_1 m_1 + \dots + v_r \bar \sigma_r m_r = 0$
				
				из того, что в свд унитарные матрицы и векторы из них составляют ОНБ, верхнее выполнено $iff$
				
				$\bar \sigma_1 m_1 + \dots +  \bar \sigma_r m_r = 0$
				
				Но сигмы не нулевые, тк это сигнулярные числа, а они будут нулю равны только начиная с r + 1
				
				Тогда m имеет вид:
				
				$\begin{pmatrix}
					0 \\
					0 \\
					\vdots \\
					m_{r + 1} \\
					m_{m} \\
				\end{pmatrix}$
			
				а это то, с чего мы начинали решение в другую сторону $\hfill \blacksquare$
				
				\item Если $P$ - ортопроектор на $ \mathrm{Im}(U_r) \implies I - P$ на $ \mathrm{Im}(U_r^\perp) =  \mathrm{ker}(A^*)$
				
				$Im(A) = Im(U_r) \implies (I - P) = (I - U_r U_r^*)$ (ну как в первом пракдз)

		\end{enumerate}

	\prob{18} Вычислите $\frac{\partial f}{\partial x}$ для следующих функционалов, где $x\in\mathbb{R}^n$. Считайте все возникающие матрицы и векторы действительными.
	\begin{enumerate}
		\item $f(x) = \|Ax - b\|_2^2$;
		
			$\|A(x + h) - b\|_2^2 = \langle A(x + h) - b, A(x + h) - b \rangle = \langle Ax, Ax \rangle + 2 \langle Ax, Ah \rangle + \langle Ah, Ah \rangle - 2 \langle Ax, b \rangle - 2 \langle Ah, b \rangle + \langle b, b \rangle = f(x) + \langle 2Ax - 2b, Ah \rangle + \langle Ah, Ah \rangle = f(x) + 2(x^T A^T - b^T )Ah + \bar o(\| h\|_2^2) $
			
			откуда 
			
			$\frac{\partial f}{\partial x} = 2A^T(AX - b)$
		
		\item $f(x) = \ln(x^\top x)$, $x\not=0$.
		
			тут сложный функционал, поэтому брать надо по очереди 
			
			знаем, что $x^T x $ это просто умный способ написать $\langle x, x \rangle $
			
			если хотим диф, то $\langle x + h, x + h \rangle = \langle x, x \rangle + 2 \langle x, h \rangle + \bar o(\|h\|_2^2) = f(x) + 2x^Th + \bar o(\|h\|_2^2)$
			
			тогда диф $2x^T$
			
			второй кусок композиции это $\frac{\partial \ln{x}}{\partial x} = \frac{1}{x}$
			
			ну теперь просто формула с семинара про свойство композиции 
			
			$\frac{df}{dx} = \frac{2x}{x^Tx}$
		
	\end{enumerate}
	\prob{20}
	Пусть $A\in\mathbb{R}^{n\times n}$ -- симметричная матрица. Пусть $X\in\mathbb{R}^{n\times p}$, $p< n$. \textbf{Указание:} при использовании правил дифференцирования необходимо указывать, на какое конкретно правило вы ссылаетесь.
	\begin{enumerate}
		\item Найдите дифференциал $f(X) = X^\top A X$.
		
			$f(x + h) - f(x) = (x + h)^T A (x + h) - x^TAx = x^T Ah + hx^TAx + h^TAh$
			
			тогда диф это $x^T Ah + hx^TAx$
			
			тут по сути ничего не юзал, чисто by def (by order of the linear f. algebra)
		
		\item Найдите дифференциал $g(X) = (X^\top X)^{-1}$.
		Напомним, что для квадратной обратимой $Y$ справедливо $d(Y^{-1})[H] = - Y^{-1} H Y^{-1}$.
		
			$dg = dg((X^\top X)^{-1})[d(X^\top X)] $ фанфект про композицию
			
			$= -(X^T X^{-1}) d(X^T X) (X^T X)^{-1} $ фанфект из условия
			
			$-(X^T X)^{-1} ((dX)^T X + X^T dX) (X^T X)^{-1}$ фанфект про произведение 
			
			kinda success
		
		\item Найдите $\frac{\partial w(X)}{\partial X}$, где $w(X) = \mathrm{Tr}(f(X) g(X))$. Считайте, что производная считается в точке $X$ с ортонормированными столбцами.
		
			$dw = d(tr(f(X) g(X))) = tr(d(f(X)g(X)))$ свойство трейса, вроде из линейности
			
			$= tr(df(X)[H]g(X) + dg(X)[H]f(X)) $ диф произведения 
			
			$= tr((H^T AX + X^T AH)g(X) - (X^T AX) (X^T X)^{-1} ((H^T X + X^T H)  (X^T X)^{-1})) = $
			
			$X^T X = I$ тк ОНБ
			
			$= tr((H^T A^T X + X^T AH)) - tr(X^T AXH^T X + X^T AXX^T H) = 2tr(X^T AH)  - tr(X^T AX H^T X) - tr(X^T AXX^TH) = 2tr(X^T AH) - tr(X^T A^T XX^T H) - tr(X^T AXX^T H) = 2tr(X^T A(I - XX^T)H)$
			
			тогда ответ $2(I - XX^T) A^TX$
		
	\end{enumerate}
	\prob{20}
	Пусть $A\in\mathbb{R}^{n\times n}$ -- симметричная невырожденная матрица.
	\begin{enumerate}
		\item Найдите матрицу $M$, такую что \[\mathrm{vec}(AX + XA) = M\,\mathrm{vec}(X)\] и укажите ее размер.
		
			$\mathrm{vec}(AX) =\begin{pmatrix}
				AX^{(1)} && \dots && AX^{(n)} \\
			\end{pmatrix}^T = (I \otimes A) \mathrm{vec}(X)$ ну это просто из связи векторизации и кронекерова произведения. 
		
			тк мы знаем, что $\mathrm{vec}(AXI) = (I^T \otimes A) \mathrm{vec}(X)$
			
			$\mathrm{vec}(AX + XA) = \mathrm{vec}(AX) + \mathrm{vec}(XA)$
			
			$A \otimes I = \begin{pmatrix}
				a_{11} I && \dots && a_{1n}I \\
				\vdots && \dots && \vdots \\
				a_{n1} I && \dots && a_{nn} I\\ 
			\end{pmatrix} \in \RR^{n^2 \times n^2}$
		
			$\mathrm{vec}(XA) = \begin{pmatrix}
				XA^{(1)} \\
				\vdots \\
				XA^{(n)} \\
			\end{pmatrix} = \begin{pmatrix}
			a_{11}X^{(1)} + && \dots && + a_{1n} X^{(n)} \\
			\vdots && \dots && \vdots \\
			a_{n1}X^{(1)}  + &&  \dots &&  + a_{nn}X^{(n)} \\
			\end{pmatrix} = (A \otimes I) \mathrm{vec}(X)$
	
			
			Итого:
			
				$\mathrm{vec}(AX) + \mathrm{vec}(XA) = (I \otimes A + A \otimes I)\mathrm{vec}(X) = M\mathrm{vec}(X) \implies M = (I \otimes A + A \otimes I)$		
				
				$M \in \RR^{n^2 \times n^2}$
		
		\item Пусть $A = S\Lambda S^{-1}$ -- собственное разложение $A$. Выразите собственные векторы и собственные значения матрицы $M$ через $S$ и $\Lambda$. \textbf{Подсказка:} вам поможет тождество $I = SS^{-1}$ и правила Кронекерова произведения.
		
			Ну нам намекнули, с чего начать
			
			$M = (I \otimes A) + (A \otimes I) = (SS^{-1}) \otimes (S\Lambda S^{-1}) + (S \Lambda S^{-1}) \otimes (SS^{-1})$
			
			вспомним, что $AB \otimes CD = (A \otimes C) (B \otimes D)$
			
			$= (S \otimes S) (S^{-1} \otimes \Lambda S^{-1}) + (S \otimes S) (\Lambda S^{-1} \otimes S^{-1}) = (S \otimes S) (S^{-1} \otimes \Lambda S^{-1} + \Lambda S^{-1} \otimes S^{-1}) =$
			
			$= (S \otimes S) (I \otimes \Lambda + \Lambda\otimes  I) (S^{-1} \otimes S^{-1})$
			
			вспомним, что $S^{-1} \otimes S^{-1} = (S \otimes S)^{-1}$, также из диагональности $I \otimes \Lambda = \Lambda \otimes I $
			
			$= (S \otimes S)(\Lambda \otimes I + I \otimes \Lambda) (S \otimes S)^{-1}$
			
			ну и в собственном разложении нам обещали, что в левой матрице живут св, а в средней значения.
			
			поэтому столбцы матрицы $(S \otimes S)$ это собственные векторы, а собственные значения выглядят как значения матрицы $(\Lambda \otimes I + I \otimes \Lambda)$
			
			Там все ребятки на диагонали и выглядят как суммы всех возможных $\lambda_i + \lambda_j$ 
			
			
					
		\item Покажите, что решение $X\in\mathbb{R}^{n\times n}$ матричного уравнения
		\[
		AX + XA = B,
		\]
		существует и единственно для любой $B\in\mathbb{R}^{n\times n}$.
		
			$\mathrm{vec}(X) = \mathrm{B}$
			
			$\displaystyle det \mathrm{vec}(X)(AX + XA) = \prod_{i \leq j} \lambda_i + \lambda_j$
			
			причем этот дет не ноль из положительной определенности, то матрица полноранговая, а значит найдется единственное решение - успех. $\hfill \blacksquare$
			
			
		
	\end{enumerate}
\end{enumerate}



\subsection*{Бонусные задачи}
\begin{enumerate}
	\item \textbf{(25 б. баллов)}.
	Пусть $A \in \mathbb{C}^{m\times n}$, $m> n$, и $B\in \mathbb{C}^{n\times n}$ --- ее подматрица максимального объема среди $n\times n$ подматриц. Докажите, что $\|AB^{-1}\|_C \leq 1$.
	\item \textbf{(30 б. баллов)}.  Пусть $P,Q$ -- ортопроекторы. Покажите, что $\|P - Q\|_2 \leq 1$.
	\item \textbf{(45 б. баллов)}. Дана матрица $A\in\mathbb{R}^{n^2\times n^2}$. Предложите, обоснуйте и запишите в виде псевдокода алгоритм решения следующей задачи:
	\[
	f(X) =  \|A - X \otimes X\|_F \to \min_{X\in\mathbb{R}^{n\times n}}
	\]
	в случае:
	\begin{enumerate}
		\item симметричной $A$,
		\item произвольной $A$.
	\end{enumerate}
	Считайте, что помимо базовых арифметических операций с матрицами, вам доступно вычисление собственного разложения матрицы \texttt{eigs} (при условии его существования), а также функции \texttt{reshape} и \texttt{transpose}.
	\textbf{Подсказка:} подумайте, как переписать  $f(X)$ с использованием $\mathrm{vec}(X)$.

\end{enumerate}


\end{document}